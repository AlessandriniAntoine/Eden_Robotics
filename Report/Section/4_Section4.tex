\section{Simulation}

\hspace{\parindent} In parallel to the design, we simulated the operation of the robot using Matlab, Simulink and python. The work explained from now on is done on the last prototype presented in the previous section. However, the principle applied has been the same throughout the project. The simultaneous work was important in order to anticipate the delays due to the manufacturing of the robot. 

\subsection{Robot kinematics}

\hspace{\parindent} The figure below shows the kinematics schema of the robot. The figure defines an \{s\} frame at the bottom, an \{e\} frame at the end effector position and a \{c\} frame at the camera position. The robot is at is home configuration. The joint are represented with the rotation (positive rotation about the axes is by the right hand rule).

\begin{figure}[ht]
    \centering
    \includegraphics[width=0.8\textwidth]{images/kinematics\_schema.png}
    \caption{Kinematics schema}
    \label{fig:mesh10}
\end{figure}

The parameters are as follows: 

\bigbreak

We can then define $M_c$ and the $M_e$ the transformation matrix ($T_{sc}$ and $T_{se}$) when the robot is at its home configuration. We can also write the screw axis $S_1,S_2,S_3,S_4$ in \{s\} and $B_1,B_2,B_3,B_4$ in \{e\}.

\subsection{URDF Format}

\subsection{Forward kinematics}
\subsubsection{Matlab simulation}

\subsubsection{Python simulation}

\subsection{Inverse kinematics}
\subsubsection{Matlab simulation}

\subsubsection{Python simulation}

\subsubsection{Torque Study}
\subsubsection{General principal}

\subsubsection{Joint results}

\subsubsection{Hardware choices}
