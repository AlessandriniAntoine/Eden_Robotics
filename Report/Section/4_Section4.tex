\section{Robot characteristics}\insertloftspace
\setcounter{figure}{0}\setcounter{table}{0}

\hspace{\parindent} In parallel to the design, we simulated the operation of the robot using Matlab, Simulink and python. The work explained from now on is done on the last prototype presented in the previous section. However, the principle applied has been the same throughout the project. The simultaneous work was important in order to anticipate the delays due to the manufacturing of the robot. 

\subsection{Kinematics}

\textbf{Definition :} Given a vector $x=[x_1 x_2 x_3]^T \in \mathbb{R}^3$, we define : 
\begin{center}
    $[x] = \begin{bmatrix}
        0 & -x_3 & x_2 \\
        x_3 & 0 & -x_1 \\
        -x_2 & x_1 & 0 \\
    \end{bmatrix}$
\end{center}

\noindent\textbf{Definition :} Given two vectors $x$ and $y$ in $\mathbb{R}^3$, we define : $x\times y = [x]\cdot y$

\bigbreak
\noindent\textbf{Definition :} For a joint, we define the pitch $h = \frac{v}{w}$ with v : the linear speed and w the angular speed

\bigbreak
\noindent\textbf{Definition :} The screw is a $6\times1$ vector that represent the angular velocity when $\dot{\theta}=1$ and the linear velocity of the origin when $\dot{\theta}=1$. $S = \begin{bmatrix} s_w\\s_v\end{bmatrix}$ with $s_v = hw-s_w\times q$ where h is the pitch and q is a point on the 

\noindent\textbf{Definition :} For a given reference frame, a screw axis S is written as 
\begin{center}
    $S=\begin{bmatrix}
        s_w\\s_v
    \end{bmatrix}$
\end{center}
where either (i) $\|s_w\|$ = 1 or (ii) $\|s_w\|$ = 0 and $\|s_v\|$ = 1. If the pitch is finite ($h$ = 0 for a pure rotation), then $s_v = hs_w-s_w\times q$ where q is a point on the axis of the screw

\bigbreak
The figure below shows the kinematics schema of the robot. The figure defines an \{s\} frame at the bottom, an \{e\} frame at the end effector position and a \{c\} frame at the camera position. The robot is at is home configuration. The joint are represented with the rotation (positive rotation about the axes is by the right hand rule).

\bigbreak 
The parameters can be found with Onshape and are listed below: 

\begin{center}
    \fcolorbox{black}{white}{
        \begin{minipage}{0.8\linewidth}
            $L_0 = 0.069m$ \hspace{3cm} $d_0 = 0m$ \hfill $h_0 = 0.06m$ \\
            $L_1 = 0.116m$ \hspace{3cm} $d_1 = 0.018m$ \\ 
            $L_2 = 0.16m$ \hspace{3.2cm} $d_2 = 0.042m$ \\ 
            $L_3 = 0.155m$ \hspace{3cm} $d_3 = 0.01413m$ \\ 
            $L_c = 0.053m$ \hspace{3cm} $d_c = 0.0105m$ \hfill $h_c = 0.0815m$ \\
            $L_e = 0.2377m$ \hfill $d_e = 0.0105$ \hfill $h_e = 5.10^{-5}m$ \\
        \end{minipage}
    }
\end{center}

\begin{figure}[ht]
    \centering
    \includegraphics[width=0.8\textwidth]{images/Section04/kinematics\_schema.png}
    \caption{Kinematics schema}
    \label{fig:mesh10}
\end{figure}
\FloatBarrier

\bigbreak
We can then define $M_c$ and the $M_e$ the transformation matrix ($T_{sc}$ and $T_{se}$) when the robot is at its home configuration. 

\bigbreak
\begin{center}
    $
    M_c = \begin{bmatrix}
        0 & 0 & 1 & -h_0-L_3-h_c\\
        0 & -1 & 0 & d_1-d_2+d_3+d_c\\
        1 & 0 & 0 & l_0+l_1+l_2+l_c\\
        0 & 0 & 0 & 1
    \end{bmatrix}
    $
    and
    $
    M_e = \begin{bmatrix}
        0 & 0 & 1 & -h_0-L_3-h_c\\
        0 & -1 & 0 & d_1-d_2+d_3+d_c\\
        1 & 0 & 0 & l_0+l_1+l_2+l_c\\
        0 & 0 & 0 & 1
    \end{bmatrix}
    $
\end{center}

\subsubsection{Base frame}

\hspace{\parindent} In this subsection we study the kinematics parameters in the base frame \{s\}. It will be the one used in the followings sections.

\bigbreak
The rotation axis $S_{w_i}$ of each joint  in \{s\} are : 
\begin{center}
    $S_{w_1} = \begin{bmatrix} 0 \\ 0 \\ -1\end{bmatrix}$,
    $S_{w_2} = \begin{bmatrix} 0 \\ 1 \\ 0\end{bmatrix}$,
    $S_{w_3} = \begin{bmatrix} 0 \\ -1 \\ 0\end{bmatrix}$,
    $S_{w_4} = \begin{bmatrix} 0 \\ -1 \\ 0\end{bmatrix}$,
\end{center}

\bigbreak
We can also write the position of each joint  $q_1,q_2,q_3,q_4,q_c,q_e$ in \{s\}. Lining up the position as columns, we get : 

\begin{center}
    $
    \begin{bmatrix}
        -h_0 & -h_0 & -h_0 & -h_0-L_3 & -h_0-L_3-h_c & -h_0-L_3-L_e  \\
        0 & d_1 & d_1-d_2 & d_1-d_2+d_3 & d_1-d_2+d_3+d_c & d_1-d_2+d_3+d_e \\
        L_0 & L_0+L_1 & L_0+L_1+L_2 & L_0+L_1+L_2 & L_0+L_1+L_2+L_c & L_0+L_1+L_2+h_e \\
    \end{bmatrix}
    $
\end{center}

\bigbreak
There are all pure rotation joint, using the position, the rotation axis and the formula define in above we can calculate the screw axis $S_1,S_2,S_3,S_4$ in \{s\}.. Lining up them as columns, we get : 

\begin{center}
    $S_{list} = 
    \begin{bmatrix}
        0 & 0 & 0 & 0 \\
        0 & 1 & -1 & -1 \\
        -1 & 0 & 0 & 0 \\
        0 & -L_0-L_1 & L_0+L_1+L_2 & L_0+L_1+L_2 \\
        -h_0 & 0 & 0 & 0 \\
        0 & -h_0 & h_0 & h_0+L_3
    \end{bmatrix}
    $
\end{center}

\subsubsection{End effector frame}

\hspace{\parindent} In this subsection we study the kinematics parameters in the end effector frame \{e\}. However, it is not the one that will be use later. 

\bigbreak
The rotation axis $S_{w_i}$ of each joint  in \{e\} are : 
\begin{center}
    $S_{w_1} = \begin{bmatrix} -1 \\ 0 \\ 0\end{bmatrix}$,
    $S_{w_2} = \begin{bmatrix} 0 \\ -1 \\ 0\end{bmatrix}$,
    $S_{w_3} = \begin{bmatrix} 0 \\ 1 \\ 0\end{bmatrix}$,
    $S_{w_4} = \begin{bmatrix} 0 \\ 1 \\ 0\end{bmatrix}$,
\end{center}

\bigbreak
We can also write the position of each joint  $q_1,q_2,q_3,q_4,q_c,q_e$ in \{e\}. Lining up the position as columns, we get : 

\begin{center}
    $
    \begin{bmatrix}
        -h_e-L_2-L_1 & -h_e-L_2 & -h_e & -h_e & -h_e+L_c & 0  \\
        d_e+d_3-d_2+d_1 & d_e+d_3-d_2 & d_e+d_3 & d_e & d_e-d_c & 0 \\
        L_e+L_3 & L_e+L_3 & L_e+L_3 & L_e & L_e-h_c & 0 \\
    \end{bmatrix}
    $
\end{center}

\bigbreak
There are all pure rotation joint, using the position, the rotation axis and the formula define in above we can calculate the screw axis $B_1,B_2,B_3,B_4$ in \{e\}.. Lining up them as columns, we get : 

\begin{center}
    $B_{list} = 
    \begin{bmatrix}
        -1 & 0 & 0 & 0 \\
        0 & -1 & 1 & 1 \\
        -1 & 0 & 0 & 0 \\
        0 & L_e+L_3 & -L_e-L_3 & -L_e \\
        -L_e-L_3 & 0 & 0 & 0 \\
        d_e+d_3-d_2+d_1 & h_e+L_2 & -h_e & -h_e
    \end{bmatrix}
    $
\end{center}

\subsection{Workspace}

\subsection{Physical characteristics}
